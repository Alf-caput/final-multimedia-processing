\section{Conclusión}
En esta práctica se ha implementado un sistema de detección de vehículos en tiempo real mediante el modelo YOLOv8, un algoritmo muy avanzado y eficiente para esta tarea.  Se han logrado procesar videos frame por frame, identificar los vehículos y sus posiciones mediante bounding boxes.  
El desarrollo incluyó varias estapas:
\begin{enumerate}
    \item Preparar entorno y herrramientas: hubo que elegir el entorno de desarrollo e instalar librerías como ultralytics y openCV.
    \item Implementaciónon: se empleó YOLO para realizar las detecciones con precisión y rapidez.  Se escribió un pipeline eficiente para procesar y guardar el vídeo.  Se filtró por clases específicas (coches y camiones).
    \item Visualización y Análisis: se incorporaron bounding boxes y etiquetas con la precisión para que fuera lo más visual y claro posible.  Se generó un archivo de salida con el vídeo procesado.
\end{enumerate}
En definitiva, se ha obtenido una detección confiable de la posición de los coches y es un modelo muy útil que podría mejorarse para ser aplicado a la vida real.
