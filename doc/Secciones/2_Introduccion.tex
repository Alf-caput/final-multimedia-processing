\section{Introducción}

\noindent
El presente trabajo tiene como objetivo principal aplicar técnicas avanzadas de procesamiento de video para la detección de vehículos, utilizando el algoritmo YOLO (\textit{You Only Look Once}), una red neuronal convolucional ampliamente reconocida por su desempeño en tareas de detección y localización de objetos en imágenes y videos.

\noindent
YOLO se destaca en el estado del arte de la detección de objetos debido a su rapidez y precisión. Este algoritmo evita el uso de \textit{pipelines} complejas, lo que le permite procesar imágenes a una velocidad de 45 cuadros por segundo (FPS). Además, logra un promedio de precisión (mAP) significativamente superior al de otros sistemas en tiempo real, consolidándose como una herramienta ideal para aplicaciones que requieren procesamiento eficiente en tiempo real.

\noindent
Entre las ventajas más relevantes de YOLO se encuentran su alta precisión en la detección de objetos, su capacidad de generalización en dominios diversos y su naturaleza de código abierto, que ha permitido una evolución constante gracias a la contribución de la comunidad investigadora. Estos factores lo posicionan como una solución robusta y versátil para tareas complejas de visión por computadora.

\noindent
El propósito específico de este trabajo es emplear YOLO para determinar la velocidad de los vehículos detectados en videos, con el fin de aplicar estos resultados a análisis posteriores en contextos reales. Dichos contextos incluyen, entre otros, el estudio del tráfico en vías altamente transitadas y el análisis de carreras automovilísticas, permitiendo una evaluación detallada y precisa.

